%% tradeoffs.tex - Resource tradeoffs for limited nondeterminism
%%
%% Copyright 2014 Jeffrey Finkelstein.
%%
%% This LaTeX markup document is made available under the terms of the Creative
%% Commons Attribution-ShareAlike 4.0 International License,
%% https://creativecommons.org/licenses/by-sa/4.0/.
\documentclass{article}

\usepackage[utf8]{inputenc}
%% This must come before hyperref.
\usepackage{amsthm}
%% This must come before hyperref.
\usepackage{thmtools}
%% This must come before complexity.
\usepackage{hyperref}
%% This is strongly recommended by biblatex.
\usepackage[english]{babel}
%% This is strongly recommended by biblatex.
\usepackage{csquotes}
\usepackage[backend=biber]{biblatex}
\usepackage{amsmath}
\usepackage{amssymb}
\usepackage{complexity}

%\hypersetup{pdftitle={Tradeoffs in limited nondeterminism}, pdfauthor={Jeffrey Finkelstein}}

\addbibresource{tradeoffs.bib}

\declaretheorem[]{theorem}
\declaretheorem[numberlike=theorem, style=definition]{definition}

\newcommand{\quant}{\mathsf{Q}}
\newcommand{\notquant}{\overline{\mathsf{Q}}}

\title{Tradeoffs in limited nondeterminism}
\author{Jeffrey Finkelstein}

\begin{document}

\maketitle

\begin{definition}
  A set of circuits $\{C_n\}_{n \in \mathbb{N}}$ is $\L$-uniform if there is a Turing machine $M$ that outputs $C_n$ on input $1^n$ while using at most $O(\log n)$ space.
\end{definition}

\begin{definition}
  An \emph{$\NC$ circuit family} is an $\L$-uniform set of circuits $\{C_n\}_{n \in \mathbb{N}}$ such that for each $n \in \mathbb{N}$, the circuit $C_n$ is composed of \textsc{and}, \textsc{or}, and \textsc{not} gates of fan-in at most two
\end{definition}

\begin{definition}
  A language $L$ is in $\NNC(\log^k n)$ if there is an $\NC$ circuit family $\{C_{n, m}\}_{n, m \in \mathbb{N}}$ such that $m \in O(\log^k n)$ and
  \begin{itemize}
  \item if $x \in L$, then there is a string $w$ such that $C_{n, m}(x, w) = 1$
  \item if $x \notin L$, then for all strings $w$, we have $C_{n, m}(x, w) = 0$
  \end{itemize}
\end{definition}

\begin{definition}[Probabilistic definition of $\NNC(\log^k n)$]
  A language $L$ is in $\NNC(\log^k n)$ if there is an $\NC$ circuit family $\{C_{n, m}\}_{n \in \mathbb{N}}$ such that $m \in O(\log^k n)$ and
  \begin{itemize}
  \item if $x \in L$, then $\Pr[C_{n, m}(x, r) = 1] > 0$
  \item if $x \notin L$, then $\Pr[C_{n, m}(x, r) = 1] = 0$
  \end{itemize}
  where the probability is over all strings $r$ of length $m$.
\end{definition}

\begin{definition}
  A language $L$ is in $\coRNC(\log^k n)$ if there is an $\NC$ circuit family $\{C_{n, m}\}_{n, m \in \mathbb{N}}$ such that for all strings $x$ of length $n$,
  \begin{itemize}
  \item if $x \in L$, then $\Pr[C_{n, m}(x, r) = 1] > \frac{1}{2}$
  \item if $x \notin L$, then $\Pr[C_{n, m}(x, r) = 1] = 0$
  \end{itemize}
  where the probability is over all strings $r$ of length $m$.
\end{definition}

\begin{definition}
  A language $L$ is in $\BPNC(\log^k n)$ if there is an $\NC$ circuit family $\{C_{n, m}\}_{n, m \in \mathbb{N}}$ such that for all strings $x$ of length $n$,
  \begin{itemize}
  \item if $x \in L$, then $\Pr[C_{n, m}(x, r) = 1] \geq \frac{2}{3}$
  \item if $x \notin L$, then $\Pr[C_{n, m}(x, r) = 1] < \frac{1}{3}$
  \end{itemize}
  where the probability is over all strings $r$ of length $m$.
\end{definition}

\begin{definition}
  A language $L$ is in $\betakSigmadNC$ if there is an $\NC$ circuit family $\{C_{n_0, \dotsc, n_d}\}_{n_0, \dotsc, n_d \in \mathbb{N}}$ such that for each $i \in \{1, \dotsc, d\}$, the value of $n_i$ is in $O(\log^k n_0)$, and for all strings $x$ of length $n_0$
  \begin{itemize}
  \item if $x \in L$, then $\exists w_1 \forall w_2 \dotsb \quant_d w_d \colon C_{n_0, \dotsc, n_d}(x, w_1, \dotsc, w_d) = 1$
  \item if $x \notin L$, then $\forall w_1 \exists w_2 \dotsb \notquant_d w_d \colon C_{n_0, \dotsc, n_d}(x, w_1, \dotsc, w_d) = 0$
  \end{itemize}
  where $\quant$ is $\forall$ if $d$ is even and $\exists$ if $d$ is odd, and $\notquant$ is the opposite.
\end{definition}

\begin{definition}
  $\betaH = \bigcup_{d \in \mathbb{N}} \betakSigmadNC$.
\end{definition}

\begin{theorem}
  $\coRNC(\log^k n) \subseteq \NNC(\log^k n)$ for all $k \in \mathbb{N}$.
\end{theorem}
\begin{proof}
  This inclusion follows immediately from the definitions.
\end{proof}

\begin{theorem}[Sipser--Gács--Lautemann]
  $\BPNC(\log^k n) \subseteq \betakSigmaTNC \cap \betakPiTNC$ for all $k \in \mathbb{N}$.
\end{theorem}
\begin{proof}[Proof sketch]
  The proof of this theorem is analagous to the proof that $\BPP \subseteq \SigmaTP \cap \PiTP$.
  The only difference is that the quantifiers need only quantify over strings of polylogarithmic length instead of strings of polynomial length.

  First, let $L$ be a language in $\BPNC(\log^k n)$.
  Use randomness-efficient error reduction to transform the algorithm with error $\frac{1}{3}$ to one with error $2^{-\log^k n}$, while still using $O(\log^k n)$ bits of randomness \autocite[Corollary~1.4]{healy08}.
  Next, construct a language $L'$ such that
  \begin{itemize}
  \item if $x \in L$, then $\exists s \forall r \colon (x, s, r) \in L'$
  \item if $x \notin L$, then $\forall s \exists r \colon (x, s, r) \notin L'$
  \end{itemize}
  where $t$ is a constant and the lengths of $s$ and $r$ are in $O(\log^k n)$.
  The string $s$ is the composition of a constant number of strings $s_i$ of length $O(\log^k n)$, each of which denotes a shift of the ``random'' string $r$.
  The language $L'$ is an $\NC$ language that encodes the problem of deciding whether the ``random'' string $r \oplus s_i$ causes the original $\BPNC(\log^k n)$ machine to accept $x$.
  This shows $\BPNC(\log^k n) \subseteq \betakSigmaTNC$.
  Since $\BPNC(\log^k n)$ is closed under complement, the result follows.
\end{proof}

%% \begin{lemma}
%%   For each $k \in \mathbb{N}$, if there is a $d \in \mathbb{N}^+$ such that $\betakSigmadNC \subseteq \betakPidNC$, then $\betakSigmadNC = \betakPidNC = \betaH$.
%% \end{lemma}
%% \begin{proof}
%%   This is a proof by induction on $d$.
%%   Let $k$ be an arbitrary integer.
%%   If $d = 1$, then 
%% \end{proof}

\begin{theorem}[Karp--Lipton]
  If $\beta_k \NC \subseteq \Ppoly$, then $\betakSigmaTNC = \betaH$.
\end{theorem}
\begin{proof}
  Everything works out as it should.
\end{proof}

\printbibliography

\end{document}
